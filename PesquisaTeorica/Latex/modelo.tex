\NeedsTeXFormat{LaTeX2e}
%-----------------------------------------------------------
\documentclass[a4paper,12pt]{monografia}
\usepackage[brazil]{babel}
\usepackage[utf8]{inputenc}
\usepackage[T1]{fontenc}
\usepackage{indentfirst}
\usepackage{hyphenat}
\usepackage{csquotes}
\usepackage{courier}
\usepackage{graphicx}
\usepackage{listings}
\usepackage{color}
\usepackage[toc,acronym]{glossaries}
\usepackage[backend=biber, style=abnt]{biblatex} 

% !!! SHARELATEX Workaround para referências ABNT (Deve-se colocar os sobrenomes em maiúscula no .bib)
%\usepackage[backend=biber, style=authortitle]{biblatex} % Descomente se estiver usando Sharelatex
%\usepackage[hidelinks]{hyperref} % Torna os links da referência clicáveis
\setlength{\bibhang}{0pt}

% === Adição do arquivo de bibliografia
\nocite{*}
\addbibresource{biblio.bib}
%\bibliography{biblio}%
%\renewcommand*{\nameyeardelim}{\addcomma\space} % estilização para citações autor/ano

% === Estilo para trechos de códigos
\definecolor{codebg}{RGB}{240,240,240}
\lstdefinestyle{codigo}{
    backgroundcolor=\color{codebg},
    keywordstyle=\color{blue},
    basicstyle=\footnotesize,
    breakatwhitespace=false,         
    breaklines=true,                 
    captionpos=b,                    
    keepspaces=true,                 
    numbers=left,                    
    numbersep=5pt,                  
    showspaces=false,                
    showstringspaces=false,
    showtabs=false,                  
    tabsize=2
}
 
\lstset{style=codigo}

% === Gerar o glossário e lista de abreviaturas (Opcional)
\makeglossaries
\newacronym{gcd}{GCD}{Greatest Common Divisor}
\newacronym{so}{SO}{Sistema Operacional}

\input{glossario}

% === Comando para imagens onde se deve destacar a fonte
\newcommand*{\captionsource}[2]{%
  \caption[{#1}]{%
    #1%
    \\\hspace{\linewidth}%
    \textbf{Fonte:} #2%
  }%
}

\begin{document}

% === Título e Dados do Autor 
\titulo{COLOQUE SEU TÍTULO AQUI}
%\subtitulo{Subtítulo} % opcional
\autor{Daniel Barlavento Gomes} \nome{Daniel} \ultimonome{Barlavento Gomes}%

% === Curso e Grau
\bacharelado \curso{Tecnologia em Análise e Desenvolvimento de Sistemas} \ano{\textbf{2017}}%
\data{DD/MM/YYYY} % data da aprovação
\cidade{Recife}
\estado{Pernambuco}

% === Informações sobre a Instituição
\instituicao{INSTITUTO FEDERAL DE EDUCAÇÃO, CIÊNCIA E TECNOLOGIA DE PERNAMBUCO} \sigla{IFPE}
\unidadeacademica{DEPARTAMENTO ACADÊMICO DE SISTEMA, PROCESSOS E CONTROLES ELETRO-ELETRÔNICO}

% === Informações obtidas na Biblioteca
%\CDU{XXX.XX} \areas{1.???  2.???}
%\npaginas{xx}  % total de páginas do trabalho

% === Nomes do Orientador, 1o. Examinador Interno e 2o. Examinador Externo
\orientador{Paulo Abadie Guedes}%
%\coorientador{Nome do Co-orientador} % opcional

\examinadorum{Examinador Interno 1}
\examinadordois{Examinador Externo 2}
%\examinadortres{Nome do Examinador 3}
%\examinadorquatro{Nome do Examinador 4}

% === Títulos do Orientador, 1o. e 2o. Examinadores
\ttorientador{Mestre}
%\ttcoorientador{Título do Co-orientador} % se digitado \coorientador

\ttexaminadorum{Doutor}
\ttexaminadordois{Doutor}
%\ttexaminadortres{Título do Examinador 3}
%\ttexaminadorquatro{Título do Examinador 4}

% === Funções do 1o. e 2o. Examinadores
\funcaoexaminadorum{Professor do Instituto Federal de Pernambuco}
\funcaoexaminadordois{Professor do Curso de Sistemas de Informações da Universidade}

\maketitle

% === Dedicatória (Opcional)
\begin{dedicatoria}
Dedico a  minha família, por todo o apoio e confiança.
\end{dedicatoria}
\maketitle

% === Agradecimentos (Opcional)
%\agradecimento{AGRADECIMENTOS}
% Digite seu texto de agradecimento aqui
%\newpage
% ou inclua um arquivo .tex como mostrado abaixo
\agradecimento{AGRADECIMENTOS}
Agradeço a meus pais Marcos Antonio da Rocha Gomes e Cristina Maria Barlavento Gomes por sempre terem me dado a liberdade da escolha. A minha avó Vanilda Negromonte Cavalcante Barlavento por todos os pudins que alimentaram corpo e alma no decorrer desta jornada. A minha amada esposa, Deborah Bezerra Monteiro pela paciência e compreensão durante todo o curso. Aos professores de TADS pela paciência e por todo o conhecimento que me foi passado. Ao meu orientador Paulo Guedes por ter aceitado me orientar e evitado que este trabalho se arrastasse para sempre. A todas as amizades feitas durante o curso, em especial a Yuri Rodrigues, Pedro Jatobá, Johnison Freitas, Edmilson Santana e Douglas Santana, que diversas vezes tiveram que segurar a "barra" e varar noites para concluir trabalhos. Ao pessoal do LIS-UFPE pela tolerância com os atrasos e ausências. A Maria Cecília C. da Mota pelo apoio e pelo computador utilizado nos testes.%
\newpage

% === Epígrafe (Opcional)
\begin{epigrafe}
``Prefiro não fazer''.\\
\hfill Herman Melville (Bartleby, o escriturário)
\end{epigrafe}

% === Inclua seu resumo em Português
%\resumo{Resumo}
% Digite seu resumo aqui
%\noindent Palavras-chave:
% ou inclua um arquivo .tex como mostrado abaixo
\resumo{RESUMO}
A crescente complexidade dos sistemas de tempo real torna necessária a utilização de técnicas e ferramentas que possibilitem aos projetistas um maior controle das aplicações desenvolvidas, tornem o desenvolvimento estruturado, possibilitem a reutilização de código e proporcionem meios para a manutenção das aplicações. Os sistemas operacionais de tempo real existem para suprir estas necessidades, a maior parte desses sistemas são proprietários e possuem um custo de licenciamento alto. Devido a necessidade de desenvolver um sistema operacional de tempo real de baixo custo diversos projetistas criaram soluções que dessem ao Linux suporte para executar aplicações de tempo real. O \textit{patch} Preempt\_RT, suportado oficialmente pelos desenvolvedores do \textit{kernel} Linux, e o RTAI, uma solução que utiliza uma arquitetura com dois \textit{kernels}, são duas soluções capazes de transformar o Linux em um sistema operacional de tempo real. Neste trabalho as duas soluções foram aplicadas a um sistema Linux que teve seu desempenho medido por meio de um conjunto de testes e os resultados avaliados, afim de verificar a real capacidade do sistema em atender os requisitos de uma aplicação de tempo real.   

\noindent Palavras-chave: Linux de Tempo Real, Sistemas de Tempo Real, Sistemas Operacionais de Tempo Real, Análise de Desempenho, Preempt\_RT, RTAI


% === Digite aqui o seu resumo em Inglês
%\resumo{Abstract}
% Digite seu resumo aqui
%\noindent Keywords:
% ou inclua um arquivo .tex como mostrado abaixo
\input{abstract}

% === Ou digite aqui o seu resumo em Francês
%\resumo{Résumé}
% Digite seu resumo aqui
%\noindent Mots-clés: 
% ou inclua um arquivo .tex como mostrado abaixo
%\input{resume}

% === Lista de figuras(Opcional), lista de tabelas(Opcional), lista de abreviações(Opcional) e sumário
%\listoffigures \thispagestyle{empty}
%\listoftables \thispagestyle{empty}
\printglossary[type=\acronymtype]
\tableofcontents

% !!! Início do Conteúdo
\pagestyle{ruledheader}

% === Hifenização
% Colocar lista de palavras que não devem ser separadas ou que não estão no dicionario português.
\hyphenation{ Hardware Software }

% === Capítulos
% A partir daqui coloque seus capítulos. Sugere-se que eles sejam inseridos com o comando \input
% Da seguinte maneira: (Devem estar na mesma pasta deste arquivo)

\chapter{INTRODUÇÃO}
\label{cap:introducao}



\section{Justificativa}


\section{Objetivos}
\subsection{Gerais}

\subsection{Específicos}
\begin{itemize}
    \item Item 1
    \item Item 2
    \item Item 3 
\end{itemize}


\chapter{ESTADO DA ARTE}
\label{cap:estadoarte}

\section{Sistemas de Tempo Real}
\subsection{Conceitos}
\subsection{Classificação}
\subsection{Algoritmos de Escalonamento}

\section{Sistemas Operacionais de Tempo Real}
\subsection{Padrão POSIX}
\subsection{Linux}
\subsection{Implementações}

\section{Patch PREEMPT-RT}

\section{RTAI}


\chapter{ANÁLISE E TESTE DOS SISTEMAS}
\label{cap:analiseemetodos}

Este capítulo apresenta as principais características utilizadas na comparação dos sistemas testados e as premissas que orientam os testes executados.

\section{Avaliação de Sistemas Operacionais de Tempo Real}
\subsection{O Ambiente de teste}

Na comparação entre diferentes sistemas operacionais, e mais especificamente de \textit{kernels} de tempo real,  é importante que a configuração do \textit{hardware} utilizado seja igual o no mínimo equivalente, isso garante que os resultados obtidos são consistentes e que não foram profundamente influenciados pelo \textit{hardware}.

O \textit{hardware} utilizado para testar as duas soluções de tempo real escolhidas foi um \textit{netbook Acer}, modelo \textit{Aspire One} D250-1023, processador com arquitetura x86, \textit{Intel Aton} N270, \textit{clock} de 1,60GHz, memória \textit{cache} L2 de 512KB, 1GB de memória DDR2-533, disco rígido de 320GB SATA.

Ambas as soluções de tempo real testadas usam como base o sistema operacional \textit{GNU/Linux}, foi escolhida a distribuição \textit{Debian} 8.8 (\textit{Jessie}) para processadores de 32 bits. A distribuição \textit{Debian} foi escolhida, dada a facilidade de se produzir um sistema com funcionalidades reduzidas, sua ampla documentação, sua grande coleção de pacotes contendo programas e bibliotecas pré compilados e por ser a base de inúmeras outras distribuições que se aplicam de servidores a sistemas embarcados.

Foi considerado de grande importância produzir \textit{kernels} com configurações idênticas, com exceção das opções específicas exigidas por cada uma das soluções, para que recursos específicos não alterassem o desempenho dos sistemas. As configurações utilizadas tiveram como ponto de partida a versão \textit{vanilla} de cada \textit{kernel}. A versão utilizada do \textit{patch PREEMPT-RT} foi a 4.4.17-rt25 publicada em 25 de agosto de 2016, aplicado sobre um \textit{kernel}, \textit{vanilla}, versão 4.4.17. A versão testada do RTAI foi a 5.0.1 publicada em 15 de maio de 2017, o \textit{patch HAL} foi aplicado em um \textit{kernel}, \textit{vanilla}, versão 4.4.43.
	
Certas funcionalidades do \textit{kernel}, como \textit{debug}, gerenciamento de energia, paginação, e consequentemente acessos a disco, podem comprometer a previsibilidade das aplicações de tempo real, para evitar estes problemas, as funcionalidades de \textit{debug} e gerenciamento e economia de energia do \textit{kernel} foram desabilitadas, os problemas relacionados a paginação e acessos a disco foram resolvidos nas próprias aplicações como veremos mais adiante. Embora esta configuração não tenha apresentado problemas no \textit{hardware} de teste, a ausência de recursos de gerenciamento de energia inviabilizou o carregamento do \textit{kernel} em outras configurações de \textit{hardware}. Como este trabalho trata do teste de soluções de tempo real que executam sobre sistemas mono processados a funcionalidade do \textit{kernel} que da suporte a \textit{SMP} foi desabilitada.
		
Para que um sistema operacional possa executar aplicações de tempo real é necessário que o sistema possua suporte a relógios com uma boa granularidade e precisão, assim as opções do \textit{kernel} relacionadas aos relógios de alta precisão foram habilitadas.
	
\subsection{Parâmetros Utilizados na Avaliação}

A avaliação e escolha de um SOTR é definida principalmente pela capacidade de suas características atenderem aos requisitos de um determinado projeto, o que pode envolver diversas variáveis que alteram o desempenho do sistema em diversas circunstâncias diferentes. Outros parâmetros relacionados a requisitos não funcionais de uma aplicação como: suporte e documentação, custo, tamanho do código, etc, corroboram com o número de possibilidades. Isso torna a comparação e escolha de um SOTR algo no mínimo confuso.

- O sistema deve apresentar latência inferior a menos de 10% do \textit{deadline} das tarefas em execução.

\section{Sobre os Testes Executados}

\subsection{Testes Preliminares}

As soluções estudadas foram submetidos a testes preliminares, que foram utilizados para identificar possíveis falhas nos processo de instalação dos sistemas como para identificar funcionalidades do \textit{kernel} que pudessem alterar o desempenho e a preempção do sistema. O principal parâmetro testado foi a latência do sistema. Os resultados obtidos com estes testes também serviram para avaliar a qualidade dos resultados obtidos com os testes desenvolvidos neste trabalho.

Estes testes foram executados por meio de ferramentas recomendadas e fornecidas pelos próprios desenvolvedores dos sistemas abordados. Foram utilizadas principalmente os programas: \textit{Latency}, para testes no \textit{RTAI} e \textit{Cyclictest}, para testes no \textit{PREEMPT-RT}. O algoritmo de medição do programa \textit{Cyclictest} foi utilizado como base para os testes desenvolvidos neste trabalho.

----Falar sobre o programa Latency ----

O programa \textit{Cyclictest} \cite{FreeSoftware} é fornecido junto a suite \textit{rt-tests}, um conjunto de ferramentas para teste de sistemas de tempo real, desenvolvidas e mantidas pelos desenvolvedores do \textit{kernel linux} e hospedada no próprio \textit{site} do projeto.

O programa \textit{Cyclictest} mede com alto grau de precisão, os resultados são fornecidos em microssegundos, a latência do sistema para um número definido de tarefas. Mostrou-se de extrema utilidade seu recurso que possibilita o rastreio de funcionalidades do \textit{kernel} que provocam aumento da latência do sistema, por meio da função \textit{FTRACER}. Este recurso foi bastante utilizado para produzir uma configuração adequada do \textit{kernel linux}.

Para que os resultados das medições obtidos com os testes sejam válidos, é preciso que os testes sejam executados diversas vezes por um período de tempo, suficiente, longo e em conjunto com alguma aplicação que sobrecarregue o sistema, com a finalidade de simular o pior cenário possível para a execução de uma aplicação de tempo real. Como os programas \textit{Cyclictest} e \textit{Latency} medem a latência do sistema, um cenário adequado de sobrecarga deve fazer uso intensivo do processador, que no pior caso deve estar com 100% de utilização de forma constante durante todo o período de execução dos testes.
A solução adotada para este problema foi a proposta por Geunsik Lim (colocar fonte). Esta abordagem, além de proporcionar o uso de 100% do processador, possui uma construção simples e independente de programas externos a maioria das distribuições \textit{Linux}.

\subsection{\textit{Banchmarks}}

\subsection{Métodos de Análise Utilizados}
\chapter{Análise dos Resultados}
\label{cap:resultados}
Os testes para \textit{Serie-PH} nos mostram intervalos de latência bastante consistentes com diferença entre os extremos suficientemente restrita, como nos mostram a tabela \ref{serie-phTabela}. Podemos dizer que quanto mais restrito o intervalo de valores mais previsível é o sistema. Na maior parte das tarefas executadas tanto no Preempt\_RT, que registrou latências menores, quanto no RTAI, intervalos mais estreitos (figura \ref{serie-phPRTvsRTAI}), com exceção da \textit{thread} 4 do teste executado no RTAI na qual fica evidente a existência de uma anomalia, e que ainda não teve sua causa identificada.

\begin{figure}[!h]
    \centering
    \includegraphics[scale=0.95]{serie-phPRTvsRTAI}
    \caption{Medidas de latência dos testes \textit{Serie-PH} - Preempt\_RT x RTAI}
    \label{serie-phPRTvsRTAI}
\end{figure}

Embora estejam distribuídos de forma  adequada, os valores máximos de latência, em alguns casos, superam 100\% do tempo de computação máximo das tarefas (tabela \ref{TcTabela}) o que pode ser um grande problema para tarefas com \textit{deadlines} na casa dos microssegundos, porém para as tarefas executadas, a soma dos valores de Latência e Tempo de Computação foram bem inferior aos deadlines definidos.

Quando adicionamos duas tarefas aperiódicas aos testes (\textit{Serie-AH}) e observamos os histogramas na figura \ref{serie-ahPRTvsRTAI}, podemos visualizar alguns comportamentos interessantes. A execução das tarefas pelo \textit{patch} Preempt\_RT a primeira vista se mostraram inalteradas, mas uma análise detalhada dos valores de latência mostram alguns pontos fora da curva e registros de latência máxima bem superiores a maioria das medições feitas nos testes da da \textit{Serie-PH}, embora os valores não tenham comprometido a execução da aplicação, a soma dos valores de latência e tempo de computação ainda foram bem inferiores ao \textit{deadline}, esse tipo de comportamento imprevisível reforça a necessidade de testes de medição de latência com a aplicação pretendida. 

\begin{figure}[!h]
    \centering
    \includegraphics[scale=0.95]{serie-ahPRTvsRTAI}
    \caption{Medidas de latência dos testes \textit{Serie-AH} - Preempt\_RT x RTAI}
    \label{serie-ahPRTvsRTAI}
\end{figure}

A coluna para os testes \textit{Serie-AH} da tabela \ref{TcTabela} nos mostra que os valores dos tempos de computação para o Preempt\_RT foram praticamente 100\% maiores que os valores medidos para os testes com a \textit{Serie-PH}, embora mais uma vez os deadlines foram respeitados. Já o RTAI não teve qualquer alteração nos valores de latência e nos valores do tempo de computação para as atividades periódicas, porém na figura \ref{serie-ahPRTvsRTAI} podemos ver um comportamento que tende a adiar a execução das tarefas aperiódicas ao longo do tempo, embora os valores tenha estado dentro de um intervalo bem definido e as curvas serem muito parecidas, podemos nos questionar se a adição de novas tarefas aperiódicas provocaria o aumento das latências destas tarefas.

A análise dos valores medidos para latências a que as tarefas de tempo real estão sujeitas e dos seus respectivos tempos de computação nos mostram que, tanto o Preempt\_RT quanto o RTAI, podem executar com segurança, tarefas de tempo real com restrições temporais na casa dos milissegundos. Para sistemas que possuam tarefas com restrições temporais na menores que 1ms e recomendada, além da execução de testes de validação dentro da própria aplicação, o projeto cuidadoso do escalonamento das tarefas, com especial atenção a atribuição da prioridades, caso seja utilizado um escalonador FIFO ou RM.

Ao compararmos os valores obtidos com os resultados apresentados na tabela \ref{CompTabela}, fornecidos por \cite{Anderson2007}, que aplicou testes semelhantes ao RTAI, e \cite{Litayem2011}, que utilizou o \textit{benchmark Cyclictest} para medir a latência no Preempt\_RT, podemos dizer que, embora tenham sido um pouco mais altos, mostram que o comportamento dos sistemas ante a heterogeneidade dos métodos de análise, testes aplicados e hardwares utilizados, esteve dentro do esperado e com uma variação de valores relativamente pequena o que seria esperado já que as principais funcionalidades de um SOTR é abstrair o comportamento do hardware e apresentar ao desenvolvedor uma camada de abstração consistente, independente da situação em que se encontra o sistema, sobre a qual possam desenvolver suas aplicações de forma previsível.

\begin{table}[!h]
\centering
\begin{tabular}{c|c|c|c|c|}
\cline{2-5} 
 & \multicolumn{2}{c|}{Preempt\_RT} & \multicolumn{2}{c|}{RTAI} \\ 
\hline 
\multicolumn{1}{ |c| }{\textit{Thread}} & Latência Máx. & Latência Mín. & Latência Máx. & Latência Mín. \\ 
\hline 
\multicolumn{1}{ |c| }{0} & 39 & 13 & 41 & 28 \\ 
\hline 
\multicolumn{1}{ |c| }{1} & 43 & 8 & 38 & 22 \\ 
\hline 
\multicolumn{1}{ |c| }{2} & 30 & 8 & 44 & 27 \\ 
\hline 
\multicolumn{1}{ |c| }{3} & 25 & 7 & 40 & 24 \\ 
\hline 
\multicolumn{1}{ |c| }{4} & 44 & 7 & 74 & 20 \\ 
\hline 
\end{tabular} 
\caption{Valores (em \si{\micro\s}) máximos e mínimos de latência obtidos nos testes \textit{Serie-PH}- Preempt\_RT x RTAI}
\label{serie-phTabela}
\end{table}

\begin{table}[!h]
\centering
\begin{tabular}{c|c|c|c|c|}
\cline{2-5} 
 & \multicolumn{2}{c|}{Preempt\_RT} & \multicolumn{2}{c|}{RTAI} \\ 
\hline 
\multicolumn{1}{ |c| }{\textit{Thread}} & Latência Máx. & Latência Mín. & Latência Máx. & Latência Mín. \\ 
\hline 
\multicolumn{1}{ |c| }{0} & 70 & 9 & 46 & 32 \\ 
\hline 
\multicolumn{1}{ |c| }{1} & 72 & 8 & 39 & 25 \\ 
\hline 
\multicolumn{1}{ |c| }{2} & 71 & 7 & 43 & 28 \\ 
\hline 
\multicolumn{1}{ |c| }{3} & 74 & 7 & 39 & 21 \\ 
\hline 
\multicolumn{1}{ |c| }{4} & 67 & 7 & 45 & 23 \\ 
\hline 
\multicolumn{1}{ |c| }{Ap. 0} & 80 & 6 & 72 & 52 \\ 
\hline 
\multicolumn{1}{ |c| }{Ap. 1} & 71 & 7 & 59 & 37 \\ 
\hline 
\end{tabular} 
\caption{Valores (em \si{\micro\s}) máximos e mínimos de latência obtidos nos testes \textit{Serie-AH}- Preempt\_RT x RTAI}
\label{serie-ahTabela}
\end{table}

\begin{table}[!th]
\centering
\begin{tabular}{c|c|c|c|c|}
\cline{2-5} 
 & \multicolumn{2}{c|}{Preempt\_RT} & \multicolumn{2}{c|}{RTAI} \\ 
\hline 
\multicolumn{1}{ |c| }{\textit{Thread}} & Tc (\textit{Serie-PH}) & Tc (\textit{Serie-AH}) & Tc (\textit{Serie-PH}) & Tc (\textit{Serie-AH}) \\ 
\hline 
\multicolumn{1}{ |c| }{0} & 18 & 70 & 19 & 21 \\ 
\hline 
\multicolumn{1}{ |c| }{1} & 38 & 81 & 45 & 45 \\ 
\hline 
\multicolumn{1}{ |c| }{2} & 39 & 83 & 37 & 36 \\ 
\hline 
\multicolumn{1}{ |c| }{3} & 27 & 70 & 28 & 20 \\ 
\hline 
\multicolumn{1}{ |c| }{4} & 39 & 79 & 43 & 40 \\ 
\hline 
\multicolumn{1}{ |c| }{Ap. 0} & - & 62 & - & 42 \\ 
\hline 
\multicolumn{1}{ |c| }{Ap. 1} & - & 56 & - & 44 \\ 
\hline 
\end{tabular} 
\caption{Valores (em \si{\micro\s}) do tempo de computação máximo obtidos nos testes \textit{Serie-PH} e \textit{Serie-AH} - Preempt\_RT x RTAI}
\label{TcTabela}
\end{table}

\begin{table}[!h]
\centering
\begin{tabular}{|c|c|c|}
\hline 
\multicolumn{2}{|c|}{RTAI \cite{Anderson2007}} & Preempt\_RT \cite{Litayem2011}\\ 
\hline 
Latência & Tc & Latência \\ 
\hline 
3 & 34 a 47 & 7 a 62 \\ 
\hline 
\end{tabular} 
\caption{Valores (em \si{\micro\s}) de latência e tempo de computação obtidos por \cite{Anderson2007} e \cite{Litayem2011}}
\label{CompTabela}
\end{table}


\chapter{CONCLUSÕES}
\label{cap:conclusoes}
Este trabalho apresentou uma análise quantitativa do desempenho de duas soluções, \textit{patch} Preempt\_RT  e RTAI, que transformam um sistemas Linux de propósito geral em um SOTR. Os resultados desta análise, os programas de teste desenvolvidos e a documentação gerada contribuem com informações valiosas para projetistas de STR e estudantes no momento de comparar outros SOTR com os sistemas estudados assim como base para a criação de ATR utilizando Linux.

A análise dos valores medidos para latências a que as tarefas de tempo real estão sujeitas e dos seus respectivos tempos de computação nos mostram que tanto o Preempt\_RT quanto o RTAI podem executar com segurança, tarefas de tempo real com restrições temporais na casa dos milissegundos e, nos casos de tarefas exclusivamente periódicas, centenas de microssegundos.

Além dos resultados obtidos com os testes, algumas conclusões sobre a utilização do \textit{patch} Preempt\_RT foram:
\begin{itemize}
    \item Facilidade de instalação
    \item Simplicidade na criação de aplicações
    \item Boa documentação, atualizada e organizada
    \item Para algumas aplicações o aparecimento de valores espúrios de latência podem comprometer seu uso
\end{itemize}

Quanto ao RTAI pode-se dizer que:
\begin{itemize}
    \item Instalar e usa-lo pela primeira vez pode ser um tormento
    \item A documentação e escassa, dispersa e desatualizada
    \item Algumas de suas funcionalidades são divulgadas pelos desenvolvedores, mas não estão documentadas
    \item Sua arquitetura lhe confere eficiência, mas pouca integração com o sistema
    \item Chamadas de sistema tornam a execução de tarefas imprevisível
    \item A maior consistência no seus valores de latência permitem produzir sistemas mais previsíveis
\end{itemize}

\section{Trabalhos Futuros}
Seria bastante desejável comprovar a eficiência do Preempt\_RT e RTAI por meio de um prova de conceito em uma aplicação prática como em um sistema de controle.
Como um dos algoritmos de escalonamento para tarefas de tempo real, o EDF é suportado tanto  pelo Preempt\_RT quanto pelo RTAI, testar a eficiência dos sistemas utilizando este algoritmo seria de grande importância.
Com a popularização de processadores com múltiplos núcleos torna inevitável o estudo do comportamento de SOTR nessas plataformas. Como também se tronaram algo popular, seria de grande interesse estudar o comportamento de um sistema Linux de tempo real em plataformas utilizadas em dispositivos embarcados baseadas em processadores ARM.
Avaliar a necessidade e a possibilidade de executar o Linux com a aplicação do \textit{patch} Preempt\_RT e do RTAI simultaneamente com o objetivo de tentar sanar deficiências de ambas as soluções.


% === Bibliografia
\newpage
%\bibliography{biblio}
\printbibliography % arquivos com as entradas bib.

% === Glossário (Opcional)
\printglossary

% === Apêndices (Opcional)
\appendix
\chapter{Principais opções alteradas na configuração do \textit{kernel}}
\label{cap:kernelconfig}

As configurações requeridas pelo Preempt\_RT e pelo RTAI estão marcadas com seus respectivos nomes. 

A opção \textit{Fully Preemptible kernel (RT)} só está disponível após a aplicação do \textit{patch} Preempt\_RT. 

A opção \textit{Interrupt pipeline} só está disponível após a aplicação do \textit{patch} HAL do RTAI. 

O arquivo de configuração utilizado tem como base a versão \textit{vanilla} dos \textit{kernels} utilizados.

\begin{itemize}
    \item Kernel 32 bits
    	\begin{itemize}
    		\item {[} {]} 64-bit kernel
    	\end{itemize}
    \item General setup > Timers subsystem
    	\begin{itemize}
    		\item {[}*{]} High Resolution Timer Support
    		\item {[}*{]} Enable Loadable module support (RTAI)
    		\item {[} {]} Module versioning support (RTAI)
    	\end{itemize}
    \item Processor type and features
    	\begin{itemize}
    		\item {[} {]} Symmetric multi-processing support
    		\item Processor family > (X) Pentium-Classic
    		\item Preemption Model > (X) Fully Preemptible kernel (RT) (Preempt\_RT)
    		\item {[}*{]} Interrupt pipeline (RTAI)
    		\item Time frequency > (X) 1000HZ
    		\item {[} {]} AMD MCE features (RTAI)
    	\end{itemize}
    \item Power Management and ACPI options
    	\begin{itemize}
    		\item {[} {]} ACPI (Advanced Configuration and Power Interface) Support
    		\item CPU Frequency scaling > {[} {]} CPU Frequency scaling
    		\item CPU Idle > {[} {]} CPU Idle PM support
    	\end{itemize}
    \item File systems > Pseudo filesystems
    	\begin{itemize}
    		\item {[}*{]} /proc file system support (RTAI)
    	\end{itemize}
    \item Kernel hacking
    	\begin{itemize}
    		\item {[} {]} Debug preemptible kernel
    		\item {[} {]} Debug the x86 FPU code
    	\end{itemize}
\end{itemize}

\chapter{Fluxograma dos programas de teste}
\label{cap:fluxograma}
\begin{figure}[!htb]
    \centering
    \includegraphics[scale=0.80]{FluxoProgramaDeTestes}
    \label{fluxograma}
\end{figure}

\chapter{\textit{Script} para inicialização do RTAI (rtai-init.bash)}
\label{cap:script}
\lstset{literate=
  {á}{{\'a}}1 {é}{{\'e}}1 {í}{{\'i}}1 {ó}{{\'o}}1 {ú}{{\'u}}1
  {Á}{{\'A}}1 {É}{{\'E}}1 {Í}{{\'I}}1 {Ó}{{\'O}}1 {Ú}{{\'U}}1
  {â}{{\^a}}1 {ê}{{\^e}}1 {î}{{\^i}}1 {ô}{{\^o}}1 {û}{{\^u}}1
  {Â}{{\^A}}1 {Ê}{{\^E}}1 {Î}{{\^I}}1 {Ô}{{\^O}}1 {Û}{{\^U}}1
  {ű}{{\H{u}}}1 {ő}{{\H{o}}}1 {ã}{{\H{a}}}1
  {ç}{{\c c}}1 {Ç}{{\c C}}1 {«}{{\guillemotleft}}1 {»}{{\guillemotright}}1
}
\lstset{language=bash}
\begin{lstlisting}[frame=single]
#!/bin/bash

# Carregando os módulos do RTAI
sudo insmod /usr/realtime/modules/rtai_hal.ko
sudo insmod /usr/realtime/modules/rtai_sched.ko
sudo insmod /usr/realtime/modules/rtai_fifos.ko
sudo insmod /usr/realtime/modules/rtai_sem.ko
sudo insmod /usr/realtime/modules/rtai_mbx.ko
sudo insmod /usr/realtime/modules/rtai_msg.ko
sudo insmod /usr/realtime/modules/rtai_shm.ko
sudo insmod /usr/realtime/modules/rtai_smi.ko
sudo insmod /usr/realtime/modules/latency_rt.ko

# Criando as variáveis para compilação
export CFLAGS=$(/usr/realtime/bin/./rtai-config --lxrt-cflags)
export LDFLAGS=$(/usr/realtime/bin/./rtai-config --lxrt-ldflags)
\end{lstlisting}

Para que as variáveis criadas pelo \textit{script} possam ser utilizadas por todos os usuários é preciso utilizar a notação ponto seguido por espaço:

\begin{lstlisting}[frame=single]
$ . rtai-init
\end{lstlisting}					

% !!! Fim do documento
\end{document}

