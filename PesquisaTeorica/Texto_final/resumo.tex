\resumo{RESUMO}
A crescente complexidade dos sistemas de tempo real torna necessária a utilização de técnicas e ferramentas que possibilitem aos projetistas um maior controle das aplicações desenvolvidas, tornem o desenvolvimento estruturado, possibilitem a reutilização de código e proporcionem meios para a manutenção das aplicações. Os sistemas operacionais de tempo real existem para suprir estas necessidades, a maior parte desses sistemas são proprietários e possuem um custo de licenciamento alto. Devido a necessidade de desenvolver um sistema operacional de tempo real de baixo custo diversos projetistas criaram soluções que dessem ao Linux suporte para executar aplicações de tempo real. O \textit{patch} Preempt\_RT, suportado oficialmente pelos desenvolvedores do \textit{kernel} Linux, e o RTAI, uma solução que utiliza uma arquitetura com dois \textit{kernels}, são duas soluções capazes de transformar o Linux em um sistema operacional de tempo real. Neste trabalho as duas soluções foram aplicadas a um sistema Linux que teve seu desempenho medido por meio de um conjunto de testes e os resultados avaliados, afim de verificar a real capacidade do sistema em atender os requisitos de uma aplicação de tempo real.   

\noindent Palavras-chave: Linux de Tempo Real, Sistemas de Tempo Real, Sistemas Operacionais de Tempo Real, Análise de Desempenho, Preempt\_RT, RTAI
