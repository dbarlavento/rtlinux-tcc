\chapter{CONCLUSÕES}
\label{cap:conclusoes}
Este trabalho apresentou uma análise quantitativa do desempenho de duas soluções, \textit{patch} Preempt\_RT  e RTAI, que transformam um sistemas Linux de propósito geral em um SOTR. Os resultados desta análise, os programas de teste desenvolvidos e a documentação gerada contribuem com informações valiosas para projetistas de STR e estudantes no momento de comparar outros SOTR com os sistemas estudados assim como base para a criação de ATR utilizando Linux.

Além dos resultados obtidos com os testes, em uma avaliação qualitativa do \textit{patch} Preempt\_RT, podemos destacar sua grande facilidade de utilização dada a sua boa documentação, mantida atualizada e organizada. As aplicações de tempo real são fáceis de construir pois utilizam as mesmas bibliotecas já suportadas e documentadas pelo Linux. A sua principal deficiência é o aparecimento de alguns valores espúrios de latência que podem comprometer o desempenho de algumas aplicações.

Quanto ao RTAI, a maior consistência nos valores de latência e tempo de execução permitem produzir sistemas mais previsíveis, porém vários problemas de suporte e compatibilidade, como documentação escassa (algumas funcionalidades nem estão documentadas), dispersa e desatualizada, arquitetura eficiente mas que lhe confere pouca integração com o Linux e a impossibilidade de executar chamadas de sistemas em tarefas de tempo real sob pena de tornar a execução do sistema imprevisível, podem tornar seu uso inviável para projetos que exijam um bom suporte ou uma aplicação que precise de uma maior integração com o Linux.

\section{Trabalhos Futuros}
Seria bastante desejável comprovar a eficiência do Preempt\_RT e RTAI por meio de um prova de conceito em uma aplicação prática como em um sistema de controle.
Como um dos algoritmos de escalonamento para tarefas de tempo real, o EDF é suportado tanto  pelo Preempt\_RT quanto pelo RTAI, testar a eficiência dos sistemas utilizando este algoritmo seria de grande importância.
Com a popularização de processadores com múltiplos núcleos torna inevitável o estudo do comportamento de SOTR nessas plataformas. Como também se tronaram algo popular, seria de grande interesse estudar o comportamento de um sistema Linux de tempo real em plataformas utilizadas em dispositivos embarcados baseadas em processadores ARM.
Avaliar a necessidade e a possibilidade de executar o Linux com a aplicação do \textit{patch} Preempt\_RT e do RTAI simultaneamente com o objetivo de tentar sanar deficiências de ambas as soluções.
