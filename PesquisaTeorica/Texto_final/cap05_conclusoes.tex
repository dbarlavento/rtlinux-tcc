\chapter{CONCLUSÕES}
\label{cap:conclusoes}
Este trabalho apresentou uma análise quantitativa do desempenho de duas soluções, \textit{patch} Preempt\_RT  e RTAI, que transformam um sistemas Linux de propósito geral em um SOTR. Os resultados desta análise, os programas de teste desenvolvidos e a documentação gerada contribuem com informações valiosas para projetistas de STR e estudantes no momento de comparar outros SOTR com os sistemas estudados assim como base para a criação de ATR utilizando Linux.

A análise dos valores medidos para latências a que as tarefas de tempo real estão sujeitas e dos seus respectivos tempos de computação nos mostram que tanto o Preempt\_RT quanto o RTAI podem executar com segurança, tarefas de tempo real com restrições temporais na casa dos milissegundos e, nos casos de tarefas exclusivamente periódicas, centenas de microssegundos.

Além dos resultados obtidos com os testes, algumas conclusões sobre a utilização do \textit{patch} Preempt\_RT foram:
\begin{itemize}
    \item Facilidade de instalação
    \item Simplicidade na criação de aplicações
    \item Boa documentação, atualizada e organizada
    \item Para algumas aplicações o aparecimento de valores espúrios de latência podem comprometer seu uso
\end{itemize}

Quanto ao RTAI pode-se dizer que:
\begin{itemize}
    \item Instalar e usa-lo pela primeira vez pode ser um tormento
    \item A documentação e escassa, dispersa e desatualizada
    \item Algumas de suas funcionalidades são divulgadas pelos desenvolvedores, mas não estão documentadas
    \item Sua arquitetura lhe confere eficiência, mas pouca integração com o sistema
    \item Chamadas de sistema tornam a execução de tarefas imprevisível
    \item A maior consistência no seus valores de latência permitem produzir sistemas mais previsíveis
\end{itemize}

\section{Trabalhos Futuros}
Seria bastante desejável comprovar a eficiência do Preempt\_RT e RTAI por meio de um prova de conceito em uma aplicação prática como em um sistema de controle.
Como um dos algoritmos de escalonamento para tarefas de tempo real, o EDF é suportado tanto  pelo Preempt\_RT quanto pelo RTAI, testar a eficiência dos sistemas utilizando este algoritmo seria de grande importância.
Com a popularização de processadores com múltiplos núcleos torna inevitável o estudo do comportamento de SOTR nessas plataformas. Como também se tronaram algo popular, seria de grande interesse estudar o comportamento de um sistema Linux de tempo real em plataformas utilizadas em dispositivos embarcados baseadas em processadores ARM.
Avaliar a necessidade e a possibilidade de executar o Linux com a aplicação do \textit{patch} Preempt\_RT e do RTAI simultaneamente com o objetivo de tentar sanar deficiências de ambas as soluções.
