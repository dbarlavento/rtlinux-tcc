\chapter{RESULTADOS}
\label{cap:resultados}
Os testes para Série-PH nos mostram intervalos de latência bastante consistentes e dentro de intervalos suficientemente restritos para a maior parte das tarefas executadas tanto no Preempt\_RT quanto no RTAI (figuras 4.1 e 4.2), com exceção da \textit{thread} 4 do teste executado no RTAI, no gráfico fica evidente a existência de uma anomalia, e que ainda não teve sua causa avaliada. Embora estejam distribuídos de forma  adequada, os valores máximos de latência, em alguns casos, superam 100\% do tempo de computação máximo das tarefas o que pode ser um grande problema para tarefas com \textit{deadlines} na casa dos microssegundos, porém para as tarefas executadas, a soma dos valores de Latência e Tempo de Computação foram bem inferior aos deadlines definidos.

Quando adicionamos duas tarefas aperiódicas aos testes (Série-AH) tivemos alguns comportamentos interessantes (figuras 4.3 e 4.4). A execução das tarefas pelo \textit{patch} Preempt\_RT a primeira vista se mostrou inalterada, mas uma análise detalhada dos valores de latência mostram alguns pontos fora da curva e registros de latência máxima bem superiores a maioria das medições, embora os valores não tenham comprometido a execução da aplicação, a soma dos valores de latência e tempo de computação ainda foram bem inferiores ao \textit{deadline}, esse tipo de comportamento reforça a necessidade de testes de medição de latência com a aplicação pretendida.

Já o RTAI teve um comportamento que parece adiar a execução das tarefas aperiódicas ao longo do tempo,  embora os valores tenha estado dentro de um intervalo definido e as curvas serem muito parecidas. Podemos nos questionar se a adição de novas tarefas aperiódicas provocaria o aumento das latências destas tarefas.

Mais uma vez os valores somados da latência com o tempo de computação das tarefas estejam bem abaixo dos valores de seus deadline, o tempo de computação das tarefas executadas pelo Preempt\_RT foram bastante elevados, enquanto no RTAI praticamente não foram alterados.
