\chapter{Metodologia}
\label{cap:projeto}

A avaliação de um SOTR é definida principalmente pela capacidade de suas características atenderem aos requisitos de um determinado projeto, o que pode envolver diversas variáveis que influenciam o desempenho do sistema em grande número circunstâncias diferentes. Características relacionadas a requisitos não funcionais de uma aplicação também  podem ter peso maior ou menor na avaliação de um SOTR, características como: suporte e reputação dos desenvolvedores, documentação, custo, integração com sistemas legados, suporte a hardwares específicos etc, corroboram com o número de fatores que podem tornam a comparação entre SOTRs um processo complexo.

As avaliações de desempenho de SOTR  mais completas, normalmente são baseadas na observação do sistema como aplicação destinada a fins específicos. Estas avaliações são difíceis de generalizar e portar para outras soluções e que possuam arquitetura de destino diferente da proposta nos testes originais. A escolha de parâmetros quantitativos que sejam comuns a maioria dos sistemas de tempo real, e que estejam diretamente relacionados a execução dos principais casos em que estes sistemas se aplicam, facilita a comparação entre as diversas soluções existentes e proporcionam uma excelente forma de avaliar SOTR.

\section{Parâmetros e premissas utilizados na Avaliação}

No sentido aplicado nesse texto o termo latência pode ser definido como: "Tempo decorrido entre o estímulo e a resposta correspondente. "Latência", em Dicionário Priberam da Língua Portuguesa [em linha], 2008-2017, https://www.priberam.pt/dlpo/Latência[consultado em 21-12-2017]. Assim como todos os sistemas reais, SOTR, estão sujeitos a latências que surgem como consequência do seu próprio funcionamento e do hardware sobre os quais executam. O conhecimento dos valores de latência e principalmente sua constância, mesmo que em um cenário de sobrecarga do sistema, são primordiais na garantia da previsibilidade e determinismo dos SOTR. O conhecimento dos valores de latência também são de vital importância na seleção de um SOTR que seja capaz de atender aos requisitos temporais de uma aplicação. 

"Isso é uma reflexão minha!" Do mesmo modo que trata outros recursos de hardware, um SOTR deve ser capaz de abstrair e gerenciar de forma adequada as latências decorrentes de suas interações com o hardware, de modo que os valores de latência apresentados ao desenvolvedor sejam o mais consistente possível dentro de um intervalo de valores bem definido, tornando o ambiente de desenvolvimento homogêneo e independente de plataforma.(???)

As causas de latência em um SOTR que mais influenciam a execução de ATR estão relacionados a:

\section{O Ambiente de Testes}
Na comparação entre diferentes SO, e mais especificamente de SOTR,  é importante que a configuração do hardware utilizado nos testes propostos seja igual ou no mínimo equivalente, isso garante que os resultados obtidos sejam consistentes e que não tenham sido influenciados por funcionalidades específicas de uma determinada configuração de hardware.

O hardware utilizado para testar as duas soluções de tempo real escolhidas foi um netbook Acer, modelo Aspire One D250-1023, processador com arquitetura x86, Intel Aton N270, clock de 1,60GHz, memória cache L2 de 512KB, 1GB de memória DDR2-533, disco rígido de 320GB SATA.

Ambas as soluções de tempo real testadas usam como base o sistema operacional Linux e a distribuição escolhida foi Debian 8.8 (Jessie) para processadores de 32 bits. A distribuição Debian foi escolhida, dada a facilidade de se produzir um sistema com funcionalidades reduzidas, sua ampla documentação, sua grande coleção de pacotes contendo programas e bibliotecas pré compilados e por ser a base de inúmeras outras distribuições que se aplicam de servidores a sistemas embarcados. 

Foi considerado de grande importância produzir \textit{kernels} com configurações idênticas, com exceção das opções específicas exigidas por cada uma das soluções, para que recursos específicos não alterassem o desempenho dos sistemas de forma a favorecer uma das soluções testadas. As configurações utilizadas tiveram como ponto de partida a versão \textit{vanilla} de cada \textit{kernel}. A versão utilizada do \textit{patch PREEMPT-RT} foi a 4.4.17-rt25 publicada em 25 de agosto de 2016, aplicado sobre um \textit{kernel}, \textit{vanilla}, versão 4.4.17. A versão testada do RTAI foi a 5.0.1 publicada em 15 de maio de 2017, o \textit{patch HAL} foi aplicado em um \textit{kernel}, \textit{vanilla}, versão 4.4.43. Vale mencionar que não existem versões do kernel que sejam suportadas por ambas as soluções simultaneamente.

\subsection{Instalação dos Sistemas e Testes Preliminares}
As soluções estudadas foram submetidos a testes preliminares, utilizados tanto para identificar possíveis falhas nos processos de instalação, como para identificar funcionalidades do \textit{kernel} que pudessem alterar o desempenho e o determinismo do sistema. Nestes testes o principal parâmetro observado foi a latência (QUAL LATÊNCIA!!!) do sistema. Embora a redução de latência (QUAL!!!), como dito anteriormente, não seja um dos principais objetivos de um SOTR, é de vital importância, junto com os outros parâmetros vistos anteriormente, que seus valores sejam conhecidos para que o sistema possa ser classificado como determinístico. Os valores de latência obtidos como resultado dos testes preliminares também serviram como referência para avaliar a qualidade e a uniformidade dos resultados obtidos com os benchmarks desenvolvidos neste trabalho.

Os testes preliminares foram executados por meio de ferramentas recomendadas e fornecidas pelos próprios desenvolvedores dos sistemas avaliados. Foram utilizados os programas: \textit{Latency}, para testes executados no \textit{RTAI} e \textit{Cyclictest}, para testes executados no \textit{kernel} com o patch \textit{PREEMPT-RT} aplicado. O algoritmo de medição do programa \textit{Cyclictest} foi utilizado como base para os testes desenvolvidos neste trabalho.

----Falar sobre o programa Latency ----

O programa \textit{Cyclictest} é fornecido junto a suíte \textit{rt-tests}, um conjunto de ferramentas para teste de sistemas de tempo real desenvolvidas e mantidas pelos desenvolvedores do \textit{kernel Linux} e hospedada no próprio repositório do \textit{kernel}.

O programa \textit{Cyclictest} mede com alto grau de precisão, os resultados são fornecidos em microssegundos, a latência do sistema para um número definido de tarefas. Mostrou-se de extrema utilidade seu recurso que possibilita o rastreio de funcionalidades do \textit{kernel} que provocam o aumento da latência do sistema, por meio da função \textit{FTRACER}. Este recurso foi utilizado para produzir uma configuração adequada do \textit{kernel linux}.
Para que os valores das medições, obtidos com os testes, sejam válidos, é preciso que os testes sejam executados diversas vezes por um período de tempo suficiente longo e que os recursos do sistema (entradas, saídas, CPU, etc) estejam sobrecarregados, reproduzindo um cenário com a pior situação possível para a execução de uma aplicação de tempo real. Como os programas \textit{Cyclictest} e \textit{Latency} medem a latência do sistema, um cenário adequado de sobrecarga é o uso intensivo do processador, que no pior caso deve estar com valores próximos de 100\% de utilização  com poucas variações durante o período de execução dos testes. A solução adotada para produzir esse cenário foi a proposta por Geusik Lin. Esta abordagem, além de proporcionar o uso de 100\% do processador, possui uma construção simples que utiliza um conjunto de instruções e programas que já se encontram pré instalados na maioria das distribuições \textit{Linux}.

\subsection{Configuração do \textit{Kernel}}  
Rever funcionalidades apontadas como vilãs da latência!

Algumas funcionalidades do \textit{kernel}, como \textit{debug}, gerenciamento de energia, paginação, e consequentemente acessos a disco, podem comprometer a previsibilidade das aplicações de tempo real, para evitar estes problemas, as funcionalidades de \textit{debug} e gerenciamento e economia de energia do \textit{kernel} foram desabilitadas, os problemas relacionados a paginação e acessos a disco foram resolvidos nas próprias aplicações como veremos mais adiante. Embora esta configuração não tenha apresentado problemas no \textit{hardware} de teste, a ausência de recursos de gerenciamento de energia inviabilizou o carregamento do \textit{kernel} em outras configurações de \textit{hardware}. Como este trabalho trata do teste de soluções de tempo real que executam sobre sistemas monoprocessados a funcionalidade do \textit{kernel} que concede suporte a \textit{SMP} foi desabilitada.

Para que um sistema operacional possa executar aplicações de tempo real é necessário que o sistema possua suporte a relógios com uma boa granularidade e precisão, assim as opções do \textit{kernel} relacionadas aos relógios de alta precisão (High Resolution Timer Support) foram habilitadas. Como sistemas de tempo real normalmente são sistemas reativos, seguindo as recomendações da configuração do kernel, a opção Clock Frequency foi configurada para 1000 Hz.

\subsection{Erros e Falhas Encontradas}

\section{\textit{Benchmarks}}
