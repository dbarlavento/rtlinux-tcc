\chapter{Metodologia}
\label{cap:projeto}

A avaliação de um SOTR é guiada principalmente pela verificação da cidade de suas características atenderem aos requisitos de um determinado projeto, o que pode envolver diversas variáveis que influenciam o desempenho do sistema em várias circunstâncias diferentes. Diferentes características relacionadas a requisitos não funcionais de uma aplicação também  podem ter peso maior ou menor na avaliação de um SOTR, características como: suporte e reputação dos desenvolvedores, documentação, custo, integração com sistemas legados, suporte a hardwares específicos etc, corroboram com o número de fatores que podem tornam a comparação entre SOTRs um processo complexo.

As avaliações de desempenho de SOTR  mais completas, normalmente são baseadas na observação do sistema como aplicação destinada a fins específicos. Estas avaliações são difíceis de generalizar e portar para outras soluções e que possuam arquitetura de destino diferente da proposta nos testes originais. Pesquisas vem sendo desenvolvidas na tentativa de criar métodos genéricos de avaliação com a escolha de parâmetros quantitativos e qualitativos que sejam comuns a maior parte dos sistemas de tempo real, e que estejam diretamente relacionados aos principais casos em que se aplicam. Estes métodos de avaliação são classificados quanto ao grau de detalhamento com que observam os sistemas testados, segundo \cite{Ghosh1994}, em: \textit{Fine-grained benchmarks}, \textit{Application-oriented benchmarks} e \textit{Simulation-based evaluations}.

Os \textit{Fine-grained benchmarks} analisam a execução de STR observando suas características de baixo nível, analisando o desempenho do conjunto hardware e software, este tipo de análise confere excelente precisão aos resultados obtidos porém requer um alto grau de conhecimento sobre o funcionamento do hardware assim como acesso ao hardware por meio de instrumentos de medição específicos. \textit{Simulation-based evaluations}, utilizam a execução de modelos, com um grau de detalhes considerado suficiente para a aplicação estudada, do sistema avaliado. Este método possui a vantagem de poder executar testes em sistemas, tant de hardware como software, que ainda não estejam completamente implementados, no entanto os resultados obtidos possuem um grau de precisão proporcional a qualidade do modelo avaliado e a quantidade de variáveis, que alteram o funcionamento do sistema, implementadas. \textit{Application-oriented benchmarks}, observam a execução de STR por meio de parâmetros de alto nível como: cumprimento de \textit{deadlines}, tempo de execução das tarefas e latências, relacionados a uma aplicação sintética com requisitos relacionados a execução semelhantes ao de aplicações reais. Este método de análise tem como principal vantagem observar o comportamento do conjunto hardware e software, porém sem que seja possível atribuir com precisão as causas dos valores medidos.

\section{Modelagem dos Teste}
Este trabalho utiliza uma abordagem de testes do tipo \textit{Application-oriented benchmarks}, concebidos para verificar a capacidade do \textit{patch} Preempt\_RT e do RTAI em transformarem um sistema Linux de propósito geral em um SOTR a partir da medida dos valores de latência, do tempo de computação das tarefas executadas e da verificação do cumprimento dos deadlines estabelecidos.

O modelo de \textit{benchmark} implementado foi baseado na solução proposta por \cite{Anderson2007} junto com o algoritmo de medição de latências do programa \textit{Cyclictest} \cite{LinuxCyclictest2017}. Em seu trabalho, \cite{Anderson2007}, propõe duas aplicações de teste que une as definições e conceitos operacionais dos \textit{benchmarks MiBench} \cite{mibench2001} e \textit{Hartstone} \cite{hartstone1992}. Em ambas as aplicações é proposto que cada uma das tarefas que as compõe execute uma função dentre as sugeridas por \textit{benchmarks MiBench} e \cite{Anderson2007} para simulação de aplicações automotivas e de controle industrial, embora vários produtos de consumo e de uso hospitalar também implementem funções semelhantes. As funções implementadas neste trabalho foram:

\begin{itemize}
   \item Multiplicação de duas matrizes 5 x 5
   \item Ordenação de 20 inteiros usando \textit{quicksort}
   \item Transformação de graus para radianos
   \item Cálculo da raiz quarada de inteiros utilizando séries de Taylor
   \item Cálculo polinomial cúbico
\end{itemize}

Dentre as duas aplicações teste desenvolvidas, a primeira utiliza a definição do \textit{benchmark Hartstone} para a Série-PH, tarefas periódicas e harmônicas, e simula o comportamento do que poderia ser um programa responsável pelo monitoramento de um conjunto de sensores com taxas de amostragem diferentes e sem a intervenção de interrupções ou do usuário. Na aplicação são implementadas cinco \textit{threads} periódicas com frequência igual a um múltiplo de todas as outras frequências maiores, no caso dos testes implementados as frequências foram as mesmas sugeridas nas definições do \textit{Hartstone}: 1Hz, 2Hz, 4Hz, 8Hz e 16Hz. Cada \textit{thread} executa uma das funções descritas anteriormente, o \textit{deadline} para a execução das tarefas foi definido como sendo igual ao seus respectivos períodos. A prioridade de cada tarefas foi a mesma para todas. 

A segunda aplicação executa, além das cinco \textit{threads} da primeira,  mais duas \textit{threads} aperiódicas, simulando aplicações que precisam responder a eventos provocados por interrupções, conforme a definição do \textit{benchmark Hartstone} para Série-AH. O intervalo de ativação das \textit{threads} aperiódicas foi gerado de forma aleatória dentro de um intervalo de 20ms a 40ms, seus \textit{deadlines} foram definidos em 20ms, e a função executada por elas foi a conversão de graus para radianos.

Além das funções listadas acima, as tarefas, tanto periódicas quanto aperiódicas, foram responsáveis pela execução das medições das suas próprias latências, tempo de computação e verificação do cumprimento de \textit{deadline}.

\section{Implementação dos Testes}
Os testes foram implementados em linguagem C, linguagem para a qual são fornecidas as bibliotecas tanto do RTAI quanto as chamadas de sistema e bibliotecas fornecidas pelo Linux, utilizadas com o Preempt\_RT.

A construção dos programas foi realizada seguindo algumas premissas sobre os sistemas que executariam as aplicações de teste. Foi levado em conta o suporte a temporizadores de alta resolução, funções com tempo de execução previsíveis, isolamento entre as tarefas, execução das aplicações em modo usuário, suporte a mecanismos que evitem paginação, cache e memória em disco. O atendimento de todas estas premissas foi feito por ambas as soluções, seja pela utilização de funções do próprio sistema Linux, seja pelo provimento de uma biblioteca específica.

Por necessidade das aplicações desenvolvidas para RTAI, que precisam utilizar funções de tempo real fornecidas por bibliotecas próprias e não podem executar chamadas de sistema dentro do código de tempo real, foram desenvolvidos quatro programas de teste: Série-PH e AH para Preempt\_RT e Série-PH AH para RTAI.

Todos os programas seguiram a mesma sequência básica de funcionamento:

\begin{enumerate}
   \item Alocação de memória
   \item Travamento das posições de memória atuais e futuras, para que permaneçam na memória \textit{RAM}
   \item Configuração das \textit{threads} como tarefas de tempo real, com a definição da prioridade e do algoritmo de escalonamento (FIFO).
   \item Execução das \textit{threads} e suas respectivas computações
\end{enumerate}

Também foram implementados nos programas de teste, um mecanismo que permite a todas as \textit{threads} de tempo real iniciarem a execução o mais próximas possível e um mecanismos para a impressão de histogramas contendo a distribuição dos resultados de medição das latências.

\section{O Ambiente de Testes}
Na comparação entre diferentes SO, e mais especificamente de SOTR,  é importante que a configuração do hardware utilizado nos testes propostos seja igual ou no mínimo equivalente, isso garante que os resultados obtidos sejam consistentes e que não tenham sido influenciados por funcionalidades específicas de uma determinada configuração de hardware.

O hardware utilizado para testar as duas soluções de tempo real escolhidas foi um netbook Acer, modelo Aspire One D250-1023, processador com arquitetura x86, Intel Aton N270, \textit{clock} de 1,60GHz, memória cache L2 de 512KB, 1GB de memória DDR2-533, disco rígido de 320GB SATA.

Antes da escolha de um hardware para a execução de qualquer uma das soluções estudadas é importante observar algumas particularidades relacionadas a possíveis conflitos entre as configurações do \textit{kernel} e o hardware que foram verificadas neste trabalho e explicadas mais adiante. 

Ambas as soluções de tempo real testadas usam como base o sistema operacional Linux e a distribuição escolhida foi Debian 8.8 (Jessie) para processadores de 32 bits. A distribuição Debian foi escolhida, dada a facilidade de se produzir um sistema com funcionalidades reduzidas, sua ampla documentação, sua grande coleção de pacotes contendo programas e bibliotecas pré compilados e por ser a base de inúmeras outras distribuições que se aplicam de servidores a sistemas embarcados. 

Foi considerado de grande importância produzir \textit{kernels} com configurações idênticas, com obvia exceção às opções específicas exigidas por cada uma das soluções, para que o máximo de recursos não alterassem o desempenho dos sistemas de forma a favorecer alguma das soluções testadas. As configurações utilizadas tiveram como ponto de partida a configuração \textit{vanilla} de cada \textit{kernel}. A versão utilizada do \textit{patch PREEMPT\_RT} foi a 4.4.17-rt25 publicada em 25 de agosto de 2016, aplicado sobre um \textit{kernel}, \textit{vanilla}, versão 4.4.17. A versão testada do RTAI foi a 5.0.1 publicada em 15 de maio de 2017, o \textit{patch HAL} foi aplicado em um \textit{kernel}, \textit{vanilla}, versão 4.4.43. Vale mencionar que não existem versões do \textit{kernel} que sejam suportadas por ambas as soluções simultaneamente.

\section{Produzindo um \textit{kernel} de tempo real}
Para produzir um sistema Linux de tempo real utilizando uma das soluções estudadas neste trabalho não é necessário nenhuma truque especial. Seguem-se todos os passos de construção de um \textit{kernel} de proposito geral (configuração, compilação, instalação dos módulos e instalação do \textit{kernel}), porém é necessária uma atenção especial durante o processo de configuração.

A construção de um SOTR baseado em Linux utilizando as duas soluções estudadas se inicia com a aplicação dos respectivos \textit{patchs} ao \textit{kernel}. Estes \textit{patchs} alteram algumas características e adicionam novas funcionalidades ao \textit{kernel}.

As principais mudanças promovidas pelo \textit{patch} Preempt\_RT é tornar o \textit{kernel} completamente preemptível, inclusive em sua região crítica, ao substituir as \textit{spin\_locks} por \textit{mutexes}, transformar os tratadores de interrupção, com exceção de interrupções vitais para o sistema como a dos temporizadores, em \textit{threads} com prioridade menor que as tarefas de tempo real, evitando assim o problema da latência provocada pela inversão de interrupção, e a implementação de uma política de herança de prioridade, minimizando o problema da inversão de prioridade.

Já o \textit{patch HAL}, fornecido junto com o RTAI, é menos invasivo e apenas adiciona ao \textit{kernel} suporte a camada de abstração de hardware \textit{ADEOS}. Esta camada de abstração permite que dois \textit{kernels} funcionem simultaneamente compartilhando o mesmo hardware. 

\subsection{Configuração do \textit{Kernel}}
Como foi dito anteriormente este processo requer uma atenção especial, pois é nesta etapa que identificaremos as funcionalidades do \textit{kernel} que provocam alterações nos valores de latência, por vezes os deixando imprevisíveis. Para auxiliar neste processo foi utilizado o programa \textit{Cyclictest} \cite{LinuxCyclictest2017}.

O \textit{Cyclictest} é um programa fornecido junto a suíte de teste \textit{rt-tests} que por sua vez é fornecida e oficialmente mantida pelo grupo de desenvolvimento do \textit{kernel}. Sua principal função é a medição do conjunto de latências provocadas pelo sistema ou por outros processos a que estão submetidas um conjunto de tarefas. \textit{Cyclictest} proporciona uma poderosa funcionalidade que detecta processos executados pelo sistema que provoquem distúrbios nos valores de latência.

As várias execuções do programa \textit{Cyclictest} mostraram que os recursos do \textit{kernel} voltados ao gerenciamento de energia foram os que mais interferiram nos valores de latência. Recomenda-se que serviços como: \textit{Advanced Configuration and Power Interface} (ACPI), CPU Frequency scaling e CPU Idle, sejam desabilitados. É importante notar que isso é uma grande restrição ao uso das soluções estudadas em sistemas alimentados por bateria.

Algumas outras funcionalidades forma desabilitadas  e habilitadas tendo como princípio exigências e características das soluções de tempo real abordadas como: temporizadores com alta resolução, carregamento de módulos etc. Outras foram alteradas afim de produzir um \textit{kernel} sem otimizações que privilegiassem algum tipo de arquitetura de hardware específica, numa tentativa de tornar o sistema o mais genérico possível.

A habilitação e remoção de recursos do \textit{kernel}, podem produzir efeitos indesejados e conflitos com algumas configurações de hardware. Alguns \textit{notebooks} não suportam a inicialização de um \textit{kernel} sem suporte a ACPI ou a incompatibilidade do RTAI com o suporte a recursos usados por alguns processadores da AMD inviabilizam o uso do sistema nesta plataforma. Infelizmente nenhuma das duas soluções possui alguma documentação sobre o hardware suportado o reforça o fato dos testes serem imprescindível.

\subsection{Instalação do RTAI, compilação e execução de programas}
Após a configuração, compilação e instalação do \textit{kernel} de tempo real, diferente do Preempt\_RT, o RTAI precisa ser configurado e instalado. O processo é semelhante a instalação de programas em Linux a partir do código fonte.

Outra exigência do RTAI é que para a compilação e execução de programas é necessário definir as variáveis \textit{CFLAGS} e \textit{LDFLAGS}, estas variáveis são utilizadas no processo de compilação e definem os locais onde se encontram as bibliotecas utilizadas na construção dos programas. Para que os programas possam ser executados os módulos do RTAI precisam ser carregados,  como os módulos são possuem um hierarquia de dependência é preciso carrega-los em uma ordem específica. Este processo foi automatizado por meio de um \textit{script} para \textit{shell}.

\section{Execução dos Testes}
Os testes forma executados sem maiores problemas e os sistemas responderam de forma adequada, sem quebras ou travamentos.

Para que os resultados pudessem servir como valores de referência e para que as reais capacidades dos sistemas testados pudessem ser avaliadas, foram executados dois procedimentos para sobrecarregar os recursos da maquina de testes conforme o proposto por \cite{Lim2017}. Um dos procedimentos, que consiste na execução infinita do comando \textit{ping} o que produz uma utilização de 100\% do processador, o outro processo foi uma compilação do \textit{kernel} Linux, que além de provocar uma sobrecarga no processamento, também provoca um intenso uso de memória \textit{RAM} e do disco rígido. Para que os valores medidos fossem consistentes os testes foram executados por um período de duas horas.

Após a execução foram plotados gráficos dos histogramas gerados, contendo os valores de latência medidos para cada uma das \textit{threads} executada, e os valores máximos de computação de cada tarefa executada foram compilados em uma tabela.  