\chapter{MATERIAIS E MÉTODOS}
\label{cap:projeto}

\section{Avaliação de Sistemas Tempo Real}
A avaliação de um SOTR é definida principalmente pela capacidade de suas características atenderem aos requisitos de um determinado projeto, o que pode envolver diversas variáveis que alteram o desempenho do sistema em diversas circunstâncias diferentes. Outros parâmetros relacionados a requisitos não funcionais de uma aplicação podem ter um peso maior ou menor na avaliação de um SOTR, como: suporte e documentação, custo, integração com sistemas legados, etc, estes corroboram com o número de fatores que tornam a comparação entre SOTR algo, no mínimo, confuso.

Avaliações de desempenho mais completas de SOTR normalmente são baseadas na avaliação do sistema como aplicações destinadas a fins específicos. Estas avaliações são difíceis de generalizar e portar para outras soluções e arquiteturas de destino diferentes da proposta original dos testes. A escolha de parâmetros quantitativos que sejam comuns a maioria dos sistemas de tempo real, e que estejam diretamente relacionados a execução dos principais casos em que estes sistemas se aplicam, facilita a comparação entre as diversas soluções existentes e proporcionam uma excelente forma de avaliação dos SOTR.

Um SO tem como principal finalidade fornecer um ambiente em que certas funcionalidades do sistema estejam ocultas ao desenvolvedor, proporcionando-lhe uma camada de abstração sobre a qual possa maximizar seu trabalho utilizando uma interface de programação mais amigável. Além desta finalidade comum aos SO, um SOTR, deve criar um ambiente de desenvolvimento previsível e determinístico qualquer que seja a carga do sistema a fim de que aplicações de tempo real possam ser executadas e seus requisitos temporais sejam respeitados. Embora o senso comum nos diga que um SOTR deva reduzir a latência entre um estímulos e suas respectivas respostas e aumentar a velocidade do sistema como um todo, provê estas características não são seu principal objetivo, embora sejam bastante desejáveis, e de até fundamentais na seleção de um SOTR. Também é importante que um SOTR proporcione meios flexíveis de implementar políticas de escalonamento e formas de controle das aplicações para que possam ser úteis em um conjunto maior de situações.

\subsection{Parâmetros de Avaliação}

\section{Testes Executados}
\subsection{O Ambiente de Testes}
Na comparação entre diferentes sistemas operacionais, e mais especificamente de kernels de tempo real,  é importante que a configuração do hardware utilizado nos testes propostos seja igual ou no mínimo equivalente, isso garante que os resultados obtidos são consistentes e que não foram profundamente influenciados pelo hardware.
O hardware utilizado para testar as duas soluções de tempo real escolhidas foi um netbook Acer, modelo Aspire One D250-1023, processador com arquitetura x86, Intel Aton N270, clock de 1,60GHz, memória cache L2 de 512KB, 1GB de memória DDR2-533, disco rígido de 320GB SATA.
Ambas as soluções de tempo real testadas usam como base o sistema operacional Linux e a distribuição escolhida foi Debian 8.8 (Jessie) para processadores de 32 bits. A distribuição Debian foi escolhida, dada a facilidade de se produzir um sistema com funcionalidades reduzidas, sua ampla documentação, sua grande coleção de pacotes contendo programas e bibliotecas pré compilados e por ser a base de inúmeras outras distribuições que se aplicam de servidores a sistemas embarcados. 

Foi considerado de grande importância produzir \textit{kernels} com configurações idênticas, com exceção das opções específicas exigidas por cada uma das soluções, para que recursos específicos não alterassem o desempenho dos sistemas de forma a favorecer uma das soluções testadas. As configurações utilizadas tiveram como ponto de partida a versão \textit{vanilla} de cada \textit{kernel}. A versão utilizada do \textit{patch PREEMPT-RT} foi a 4.4.17-rt25 publicada em 25 de agosto de 2016, aplicado sobre um \textit{kernel}, \textit{vanilla}, versão 4.4.17. A versão testada do RTAI foi a 5.0.1 publicada em 15 de maio de 2017, o \textit{patch HAL} foi aplicado em um \textit{kernel}, \textit{vanilla}, versão 4.4.43. Vale mencionar que não existem versões do kernel que sejam suportadas por ambas as soluções.

\subsection{Testes Preliminares}
As soluções estudadas foram submetidos a testes preliminares, utilizados tanto para identificar possíveis falhas nos processo de instalação dos sistemas como para identificar funcionalidades do \textit{kernel} que pudessem alterar o desempenho e a preempção do sistema. Nestes testes o principal parâmetro observado foi a latência do sistema. Embora a redução de latência, como dito anteriormente, não seja um dos principais objetivos de um SOTR, é de vital importância, junto com outros parâmetros, que seus valores sejam conhecidos e mantidos constante para que o sistema seja considerado determinístico.

Os valores de latência obtidos como resultado dos testes preliminares também serviram como referência para avaliar a qualidade e a uniformidade dos resultados obtidos com os benchmarks desenvolvidos neste trabalho.

Os testes preliminares foram executados por meio de ferramentas recomendadas e fornecidas pelos próprios desenvolvedores dos sistemas avaliados. Foram utilizados os programas: \textit{Latency}, para testes executados no \textit{RTAI} e \textit{Cyclictest}, para testes executados no \textit{kernel} com o patch \textit{PREEMPT-RT} aplicado. O algoritmo de medição do programa \textit{Cyclictest} foi utilizado como base para os testes desenvolvidos neste trabalho.

----Falar sobre o programa Latency ----

O programa \textit{Cyclictest} é fornecido junto a suíte \textit{rt-tests}, um conjunto de ferramentas para teste de sistemas de tempo real desenvolvidas e mantidas pelos desenvolvedores do \textit{kernel Linux} e hospedada no próprio repositório do \textit{kernel}.
O programa \textit{Cyclictest} mede com alto grau de precisão, os resultados são fornecidos em microssegundos, a latência do sistema para um número definido de tarefas. Mostrou-se de extrema utilidade seu recurso que possibilita o rastreio de funcionalidades do \textit{kernel} que provocam o aumento da latência do sistema, por meio da função \textit{FTRACER}. Este recurso foi utilizado para produzir uma configuração adequada do \textit{kernel linux}.
Para que os valores das medições, obtidos com os testes, sejam válidos, é preciso que os testes sejam executados diversas vezes por um período de tempo suficiente longo e que os recursos do sistema (entradas, saídas, CPU, etc) estejam sobrecarregados, reproduzindo um cenário com a pior situação possível para a execução de uma aplicação de tempo real. Como os programas \textit{Cyclictest} e \textit{Latency} medem a latência do sistema, um cenário adequado de sobrecarga é o uso intensivo do processador, que no pior caso deve estar com valores próximos de 100\% de utilização  com poucas variações durante o período de execução dos testes.
A solução adotada para deste cenário foi a proposta por Geusik Lin. Esta abordagem, além de proporcionar o uso de 100\% do processador, possui uma construção simples que utiliza um conjunto de instruções e programas que já se encontram pré instalados na maioria das distribuições \textit{Linux}.

\subsection{Configuração do \textit{kernel}}  
Rever funcionalidades apontadas como vilãs da latência!

Algumas funcionalidades do \textit{kernel}, como \textit{debug}, gerenciamento de energia, paginação, e consequentemente acessos a disco, podem comprometer a previsibilidade das aplicações de tempo real, para evitar estes problemas, as funcionalidades de \textit{debug} e gerenciamento e economia de energia do \textit{kernel} foram desabilitadas, os problemas relacionados a paginação e acessos a disco foram resolvidos nas próprias aplicações como veremos mais adiante. Embora esta configuração não tenha apresentado problemas no \textit{hardware} de teste, a ausência de recursos de gerenciamento de energia inviabilizou o carregamento do \textit{kernel} em outras configurações de \textit{hardware}. Como este trabalho trata do teste de soluções de tempo real que executam sobre sistemas monoprocessados a funcionalidade do \textit{kernel} que concede suporte a \textit{SMP} foi desabilitada.
Para que um sistema operacional possa executar aplicações de tempo real é necessário que o sistema possua suporte a relógios com uma boa granularidade e precisão, assim as opções do \textit{kernel} relacionadas aos relógios de alta precisão (High Resolution Timer Support) foram habilitadas. Como sistemas de tempo real normalmente são sistemas reativos, seguindo as recomendações da configuração do kernel, a opção Clock Frequency foi configurada para 1000 Hz.

\subsection{\textit{Benchmarks}}
